\chapter{Conclusioni}
Si conclude qui la relazione relativa all'analisi delle sollecitazioni su una trave e due pilastri. I risultati di maggior interresse sono sicuramente i diagrammi degli inviluppi delle sollecitazioni agli Stati Limite Ultimi (figure~\ref{fig:bendingMomentEnvelope_slu}, \ref{fig:shearEnvelope_slu},  \ref{fig:P27axialLoad_slu}, \ref{fig:P36axialLoad_slu}) nonché le tabelle contenenti i valori massimi delle azioni interne della trave in esame (tabelle~\ref{tab:max_min_bendingMomentEnvelope_slu} e \ref{tab:shearEnvelope_slu}).

Sono comunque rilevanti anche i diagrammi degli Stati Limite di Esercizio per quanto riguarda la fessurazione degli elementi strutturali e gli spostamenti (vedi paragrafi relativi). 

In definitiva i valori e i diagrammi riportati saranno utili per il dimensionamento degli elementi strutturali, argomento non trattato in questo corso.
