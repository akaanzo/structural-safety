\chapter{Pilastri}\label{chap:columns}
Come specificato nel capitolo~\ref{chap:intro}, i pilastri da analizzare sono il $P27$ e il $P36$. Il primo è un pilastro centrale, il cui calcolo è relativamente semplice. Al contrario, il secondo è un pilastro d'angolo che divide l'ambiente interno da quello esterno. Il calcolo di quest'ultimo risulterà leggermente più complicato in termini di analisi dei carichi e di combinazioni. 

Per entrambi i pilastri si ipotizzerà l'azione interna di sola compressione (si rimuove quindi l'ipotesi di presso-flessione) per cui idealmente i carichi sono applicati nel nocciolo centrale d'inerzia. Questo rende l'analisi semplificata.

Inoltre, come consigliato dal docente, per il calcolo delle combinazioni si farà una combinazione lineare (somma) delle combinazioni di carico massime (o minime, a seconda dei casi) di ogni piano; questo nell'ipotesi che il carico totale su ogni piano sia massimizzato.

\contentinput{sections/p27}
\contentinput{sections/p36}
