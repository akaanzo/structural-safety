\section{Analisi dei Carichi}\label{sec:loads}
 La trave in esame, che si trova al secondo impalcato (piano primo), è una trave perimetrale; le dimensioni della sezione, perciò, sono $30\times 50\,\si{cm}$ rispettivamente per la base e per l'altezza della trave.

\begin{figure}
 \centering
 \begin{tikzpicture}[scale=.7]
  \draw [thick, pattern=north east lines](-1.5, -2.5) rectangle (1.5, 2.5);
  \draw [|-|] (-1.5, -3.5) -- (1.5, -3.5) node at (0, -3.5) [anchor=south]{$30\,\si{cm}$} ;
  \draw [|-|] (-2, -2.5) -- (-2, 2.5) node at (-2, 0) [rotate=90, anchor=south]{$50\,\si{cm}$} ;
 \end{tikzpicture}
 \caption{Sezione della trave in cemento armato}
 \label{fig:beamSec}
\end{figure}

Come si può notare dalla figura~\ref{fig:pianoPrimo}, essendo la trave perimetrale, divide due ambienti: l'ambiente interno adibito ad uffici aperti al pubblico e il terrazzo esterno. Sarà, quindi,  necessario calcolare i carichi separatamente per la zona interna e per la zona esterna.

Nei seguenti paragrafi verranno elencati i carichi agenti sulla trave.

\subsection{Carichi permanenti strutturali}
Per quanto riguarda i carichi permanenti strutturali, questi si identificano nel \emph{peso proprio della trave} e nel \emph{peso proprio del solaio strutturale}; entrambi sono stati definiti nel capitolo~\ref{chap:intro}, rispettivamente pari a
\begin{equation*}
 G_{1k, t} = \gamma_{cls}\cdot B\cdot H = 25\,\dfrac{kN}{m^3} \cdot 0.30\,\si{m}\cdot 0.50\,\si{m} = 3.75\,\dfrac{kN}{m}
\end{equation*}
per la trave, dove $B$ è la base e $H$ è l'altezza della sezione. Mentre il peso del solaio strutturale è
\begin{equation*}
 g_{1k, solaio} = 3.20\,\dfrac{kN}{m^2}
\end{equation*}

\subsection{Carichi permanenti non strutturali}
Al contrario dei carichi strutturali, che in questo caso sono i medesimi e per il solaio interno e per il terrazzo, i carichi non strutturali si distinguono in funzione dell'ambiente.

\subsubsection*{Solaio interno}

\begin{align*}
 g_{2k, sottofondo} &= \gamma_{cls, all}\cdot s_{sottofondo} = 16\,\dfrac{kN}{m^3}\cdot 0.08\,\si{m} = 1.28\,\dfrac{kN}{m^2}\\
 g_{2k, massetto} &= \gamma_{massetto}\cdot
 s_{massetto} = 24\,\dfrac{kN}{m^3}\cdot 0.06\,\si{m} = 1.44\,\dfrac{kN}{m^2}\\
 g_{2k, pavimento} &= 0.50\,\dfrac{kN}{m^2}\\
 g_{2k, intonaco} &= \gamma_{intonaco}\cdot s_{intonaco} = 20\,\dfrac{kN}{m^3}\cdot 0.01\,\si{m} = 0.20\,\dfrac{kN}{m^2}
\end{align*}

Discorso diverso per quanto riguarda le divisorie interne. Il peso per unità di lunghezza della tramezza in laterizio è di 
\begin{align*}
 G_{2, div} &= \gamma_{div} \cdot s_{div} \cdot H_{div} + 2 \cdot \gamma_{intonaco} \cdot s_{intonaco} \cdot h_{div} =\\ &= 8.00\,\dfrac{kN}{m^3}\cdot 0.08\,\si{m} \cdot (3.10 - 0.25)\,\si{m} + 20\,\dfrac{kN}{m^3}\cdot 0.01\,\si{m} \cdot (3.10 - 0.25 - 0.01)\,\si{m}=\\ &= 2.96\,\dfrac{kN}{m}
 \end{align*}
 
 La normativa consente di convertire il carico per unità di lunghezza  delle divisorie in un carico uniformemente distribuito su un'area. Per elementi divisori con $2<G_2 \geq 3\,kN/_m$ si ha
 \[
  g_{2k, div} = 1.20\,\dfrac{kN}{m^2}
 \]
 
 In definitiva, per il solaio interno, i carichi permanenti non strutturali valgono
 \begin{align*}
	g_{2k, interno} &= g_{2k, sottofondo} + g_{2k, massetto} + g_{2k, pavimento} + g_{2k, intonaco} + g_{2k, div}=\\ &= (1.28 + 1.44 + 0.50 + 0.20 + 1.20)\,\dfrac{kN}{m^2} =\\&=
	4.62\,\dfrac{kN}{m^2}
 \end{align*}
 
 \subsubsection*{Solaio terrazza}
 Il solaio della terrazza si distingue da quello interno principalmente per la pavimentazione e per la presenza dell'solante termico.

\begin{align*}
 g_{2k, isolante} &= \gamma_{isolante}\cdot s_{isolante} = 0.50\,\dfrac{kN}{m^3}\cdot 0.15\,\si{m} = 0.075\,\dfrac{kN}{m^2}\\
 g_{2k, massetto} &= \gamma_{massetto}\cdot s_{massetto} = 24\,\dfrac{kN}{m^3}\cdot 0.06\,\si{m} = 1.44\,\dfrac{kN}{m^2}\\
 g_{2k, pavimento} &= 0.50\,\dfrac{kN}{m^2}\\
 g_{2k, intonaco} &= \gamma_{intonaco}\cdot s_{intonaco} = 20\,\dfrac{kN}{m^3}\cdot 0.01\,\si{m} = 0.20\,\dfrac{kN}{m^2}
\end{align*}

Sommando i contributi dei singoli carichi distribuiti si ottiene il valore finale per la terrazza
 \begin{align*}
	g_{2k, terrazza} &= g_{2k, isolante} + g_{2k, massetto} + g_{2k, pavimento} + g_{2k, intonaco} =\\ &= (0.075 + 1.44 + 0.50 + 0.20)\,\dfrac{kN}{m^2} =\\&=
	2.215\,\dfrac{kN}{m^2}
 \end{align*}

\subsubsection*{Tamponamenti perimetrali}
Il peso della muratura perimetrale ricade nei carichi permanenti non strutturali. Secondo quanto descritto nel capitolo~\ref{chap:intro}, la stratigrafia della parete è quella descritta in figura~\ref{fig:stratigrafiaTamponamento}

\begin{figure}
 \centering
 \begin{tikzpicture}[scale=.5]
  \draw (0,0) rectangle (3, 10);
    \foreach \y in {0, .4, ..., 10}
	\draw (0, \y) -- (3, \y);
	
	\foreach \x in {0, .4, ..., 3}
		\draw (\x, 0) -- (\x, 10);

\node at (1.5, 5) [fill=white, rotate=90] {$laterizio$};

 \draw [fill=black] (3,0) rectangle (3.1, 10);
 \draw [->, thin] (3.05, 5) -- +(1, 0) node [anchor = west] {$intonaco$};
 
 \draw[pattern=north east lines] (0,0) rectangle (-1.2, 10);
 \node at (-.6, 5) [fill=white, rotate=90] {$cappotto$};
 
 \draw[|-|, thin] (-1.2, -2) -- (0, -2);\draw[-|, thin] (0, -2) -- (3, -2); \draw[-|, thin](3, -2) -- (3.1, -2);
 \node at (-.6, -1.8) [anchor=south] {$12\si{cm}$};
  \node at (1.5, -1.8) [anchor=south] {$30\si{cm}$};
   \node at (3.2, -1.8) [anchor=south] {$1\si{cm}$};
 \end{tikzpicture}
 \caption{Stratigrafia tamponamento perimetrale}
 \label{fig:stratigrafiaTamponamento}
\end{figure}

Il carico, che risulterà essere distribuito lungo una linea e gravante direttamente sulla trave, ha valore
\begin{align*}
 G_{2k, tamp} &= \gamma_{tamp} \cdot s_{tamp} \cdot h_{piano} + \gamma_{cappotto} \cdot s_{cappotto} \cdot h_{piano} + \gamma_{intonaco} \cdot s_{intonaco} \cdot h_{piano} =\\
 &= \left(10\,\dfrac{kN}{m^3}\cdot 0.30\,\si{m} + 0.20\,\dfrac{kN}{m^3}\cdot 0.12\,\si{m} + 20\,\dfrac{kN}{m^3}\cdot 0.01\,\si{m}\right)\cdot 2.85\,\si{m} = 9.20\,\dfrac{kN}{m}
\end{align*}


\subsection{Carichi variabili}
Come già detto in precedenza, la trave è una trave perimetrale che divide due ambienti: un ambiente interno adibito a uffici aperti al pubblico e una terrazza esterna soggetta alle azioni meteorologiche. Si ha la necessità, anche in questo caso, di suddividere i carichi variabili dei due differenti ambienti.

\subsubsection*{Solaio interno}
Il solaio interno deve essere progettato per sostenere degli uffici aperti al pubblico. Seguendo quanto descritto nelle \emph{NTC 2018}, in particolare nella \textbf{Tab. 3.1.II}, la categoria in cui si ricade è la \emph{B2}. 

Si deduce che il valore del sovraccarico è di 
\[
 q_{k, CAT.\,B2} = 3\,\dfrac{kN}{m^2}
\]

\subsubsection*{Terrazza}
Sul solaio della terrazza si deve tener conto e del carico variabile della categoria definita sopra (con riferimento al valore per \emph{balconi} e \emph{ballatoi}) e delle azioni dei fenomeni meteorologici quali neve e vento. In ordine si trova
\[
 q_{k, CAT.B}^{balconi} = 4\,\dfrac{kN}{m^2}
\]

Il carico dovuto alla neve è calcolato seguendo quanto scritto nel \textbf{capitolo 3.4} delle \emph{NTC 2018}. In particolare, la \emph{[3.4.1]} definisce il carico provocato dalla neve sulla copertura
\[
 q_s = q_{sk}\cdot \mu_i \cdot C_E \cdot C_t
\]

L'edificio si trova in \emph{Provincia di Trento}, il che ricade nella zona di carico neve \textbf{'I - Alpina'} che gli conferisce un carico di neve al suolo pari a
\begin{align*}
 q_{sk} = 
 \begin{cases}
    1.50\,\dfrac{kN}{m^2}\quad &\text{se } a_s < 200\,\si{m}\\\\
    1.39\,\left[ 1 + \left(\dfrac{a_s}{728}\right)^2\right]\,\left[\dfrac{kN}{m^2}\right]\quad &\text{se } a_s > 200\,\si{m}
 \end{cases}
\end{align*}

L'altitutide è $788\,\si{m} > 200\,\si{m}$ per cui
\[
 q_{sk} = 1.39\,\left[ 1 + \left(\dfrac{788}{728}\right)^2\right] = 3.02\,\dfrac{kN}{m^2}
\]

Poiché non sono presenti informazioni riguardo la localizzazione e quindi l'esposizione dell'edificio al vento che produrrebbe una rimozione della neve sulla copertura, si considera la topografia dell'area \emph{normale}, cioè si assume
\[
 C_E = 1
\]

Inoltre, non avendo documenti che descrivono il fenomeno di scioglimento della neve causato dal riscaldamento prodotto all'interno dell'edifico, si assume cautelativamente il coefficiente termico unitario
\[C_t = 1.\]





