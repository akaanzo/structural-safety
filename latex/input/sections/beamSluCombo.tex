\section{Combinazione agli SLU}
Il \textbf{capitolo 2.5.3} delle \emph{NTC 2018} descrive, ai fini delle verifiche agli stati limite, le varie combinazioni delle azioni. 

La combinazione fondamentale, impiegata per il calcolo agli \emph{stati limite ultimi} è descritta dalla \emph{[2.5.1]} ed è così formulata
\[
	\gamma_{G1}\cdot G_{1k} + \gamma_{G2}\cdot G_{2k} + \gamma_{Q1} \cdot Q_{1k} + \gamma_{Q2}\cdot \psi_{02}\cdot Q_{2k} + \gamma_{Q3}\cdot \psi_{03}\cdot Q_{3k} + \dots
\]

Si riportano di seguito il calcolo dei carichi distribuiti secondo la formulazione agli SLU, in funzione dei tratti di trave in esame.

Ricordando che si sta trattando una trave di bordo, saranno necessarie diverse combinazioni di carico per valutare la massima azione sulla trave. È comunque evidente che sovraccarichi di minor rilevanza, in termini di azione, daranno un contributo molto minore rispetto a sovraccarichi più elevati. Verranno tralasciati, pertanto, le combinazioni ove questi sovraccarichi risultano essere principali.

\subsubsection*{Trave $P13\div P16$}
Dalla tabella~\ref{tab:azioniTrave} si può vedere che i carichi che generano un'azione di maggior rilievo sono quelli di categoria (B) della terrazza e il carico da neve. Verranno, perciò, calcolate due sole combinazioni.

\paragraph{Combinazione 1: neve principale}
Nella combinazione per cui l'azione della neve è principale, le altre azioni saranno ridotte di un fattore che dipende dalla categoria di cui fanno parte.

\begin{align*}
	Q_{11}^{max} &=\gamma_{G1_{SF}}\cdot\left( G_{1k, interno} + G_{1k, terrazza} + G_{1k, trave} \right) +\\
	&+\gamma_{G2_{SF}}\cdot\left( G_{2k, interno} + G_{2k, terrazza} + G_{2k, tamponamento} \right) +\\
	&+\gamma_{Q_{SF}}\cdot (Q_{neve} + \psi_{0,CAT.B2,int}\cdot Q_{CAT.B2,int} + \psi_{0,CAT.B2,terr}\cdot Q_{CAT.B2,terr} +\\
	&+\psi_{0,vento}\cdot Q_{vento}) =\\
	&= 1.3\cdot(8+11.52 + 3.75)	+1.5\cdot(11.55+7.98 + 9.20) + 1.5\cdot(15.01) +\\
	&+1.5\cdot(0.7\cdot7.5+ 0.7\cdot 14.40 + 0.6\cdot 0.5) = \\
	&= 	119.306\,kN
\end{align*}






\paragraph{Combinazione 2: categoria B (terrazza) principale}

\begin{align*}
	Q_{12}^{max} &= 1.3\cdot(8+11.52 + 3.75) 
	+1.5\cdot(11.55+7.98 + 9.20) +\\
	&+1.5\cdot(0.5\cdot 15.01 + 0.7\cdot7.5+ 14.40 + 0.6\cdot 0.5) = \\
	&= 	114.345\,kN
\end{align*}

\paragraph{Combinazione 3: vento principale}
Per valutare il minimo, viene considerata l'azione del vento che 'alleggerisce' la trave, quella cioè con valore negativo.

\begin{align*}
	Q_{13}^{min} &=\gamma_{G1_{F}}\cdot\left( G_{1k, interno} + G_{1k, terrazza} + G_{1k, trave} \right) +\\
	&+\gamma_{G2_{F}}\cdot\left( G_{2k, interno} + G_{2k, terrazza} + G_{2k, tamponamento} \right) +\\
	&+\gamma_{Q_{F}}\cdot (\psi_{0,neve}\cdot Q_{neve} + \psi_{0,CAT.B2,int}\cdot Q_{CAT.B2,int} + \psi_{0,CAT.B2,terr}\cdot Q_{CAT.B2,terr}) +\\
	&+ \gamma_{vento_{SF}}\cdot Q_{vento} =\\
	&= 1\cdot(8+11.52 + 3.75) +0.8 \cdot(11.55+7.98 + 9.20) +\\
	&+ 0\cdot(0.5\cdot 15.01 + 0.7\cdot7.5+ 0.7\cdot 10.80) - 1.5\cdot 0.5 = \\
	&= 	45.50\,kN
\end{align*}


